% Options for packages loaded elsewhere
\PassOptionsToPackage{unicode}{hyperref}
\PassOptionsToPackage{hyphens}{url}
%


\PassOptionsToPackage{table}{xcolor}

\documentclass[
  10pt,
  letterpaper,
]{article}

\usepackage{amsmath,amssymb}
\usepackage{iftex}
\ifPDFTeX
  \usepackage[T1]{fontenc}
  \usepackage[utf8]{inputenc}
  \usepackage{textcomp} % provide euro and other symbols
\else % if luatex or xetex
  \usepackage{unicode-math}
  \defaultfontfeatures{Scale=MatchLowercase}
  \defaultfontfeatures[\rmfamily]{Ligatures=TeX,Scale=1}
\fi
\usepackage{lmodern}
\ifPDFTeX\else  
    % xetex/luatex font selection
\fi
% Use upquote if available, for straight quotes in verbatim environments
\IfFileExists{upquote.sty}{\usepackage{upquote}}{}
\IfFileExists{microtype.sty}{% use microtype if available
  \usepackage[]{microtype}
  \UseMicrotypeSet[protrusion]{basicmath} % disable protrusion for tt fonts
}{}
\makeatletter
\@ifundefined{KOMAClassName}{% if non-KOMA class
  \IfFileExists{parskip.sty}{%
    \usepackage{parskip}
  }{% else
    \setlength{\parindent}{0pt}
    \setlength{\parskip}{6pt plus 2pt minus 1pt}}
}{% if KOMA class
  \KOMAoptions{parskip=half}}
\makeatother
\usepackage{xcolor}
\usepackage[top=0.85in,left=2.75in,footskip=0.75in]{geometry}
\setlength{\emergencystretch}{3em} % prevent overfull lines
\setcounter{secnumdepth}{-\maxdimen} % remove section numbering

\usepackage{color}
\usepackage{fancyvrb}
\newcommand{\VerbBar}{|}
\newcommand{\VERB}{\Verb[commandchars=\\\{\}]}
\DefineVerbatimEnvironment{Highlighting}{Verbatim}{commandchars=\\\{\}}
% Add ',fontsize=\small' for more characters per line
\usepackage{framed}
\definecolor{shadecolor}{RGB}{241,243,245}
\newenvironment{Shaded}{\begin{snugshade}}{\end{snugshade}}
\newcommand{\AlertTok}[1]{\textcolor[rgb]{0.68,0.00,0.00}{#1}}
\newcommand{\AnnotationTok}[1]{\textcolor[rgb]{0.37,0.37,0.37}{#1}}
\newcommand{\AttributeTok}[1]{\textcolor[rgb]{0.40,0.45,0.13}{#1}}
\newcommand{\BaseNTok}[1]{\textcolor[rgb]{0.68,0.00,0.00}{#1}}
\newcommand{\BuiltInTok}[1]{\textcolor[rgb]{0.00,0.23,0.31}{#1}}
\newcommand{\CharTok}[1]{\textcolor[rgb]{0.13,0.47,0.30}{#1}}
\newcommand{\CommentTok}[1]{\textcolor[rgb]{0.37,0.37,0.37}{#1}}
\newcommand{\CommentVarTok}[1]{\textcolor[rgb]{0.37,0.37,0.37}{\textit{#1}}}
\newcommand{\ConstantTok}[1]{\textcolor[rgb]{0.56,0.35,0.01}{#1}}
\newcommand{\ControlFlowTok}[1]{\textcolor[rgb]{0.00,0.23,0.31}{#1}}
\newcommand{\DataTypeTok}[1]{\textcolor[rgb]{0.68,0.00,0.00}{#1}}
\newcommand{\DecValTok}[1]{\textcolor[rgb]{0.68,0.00,0.00}{#1}}
\newcommand{\DocumentationTok}[1]{\textcolor[rgb]{0.37,0.37,0.37}{\textit{#1}}}
\newcommand{\ErrorTok}[1]{\textcolor[rgb]{0.68,0.00,0.00}{#1}}
\newcommand{\ExtensionTok}[1]{\textcolor[rgb]{0.00,0.23,0.31}{#1}}
\newcommand{\FloatTok}[1]{\textcolor[rgb]{0.68,0.00,0.00}{#1}}
\newcommand{\FunctionTok}[1]{\textcolor[rgb]{0.28,0.35,0.67}{#1}}
\newcommand{\ImportTok}[1]{\textcolor[rgb]{0.00,0.46,0.62}{#1}}
\newcommand{\InformationTok}[1]{\textcolor[rgb]{0.37,0.37,0.37}{#1}}
\newcommand{\KeywordTok}[1]{\textcolor[rgb]{0.00,0.23,0.31}{#1}}
\newcommand{\NormalTok}[1]{\textcolor[rgb]{0.00,0.23,0.31}{#1}}
\newcommand{\OperatorTok}[1]{\textcolor[rgb]{0.37,0.37,0.37}{#1}}
\newcommand{\OtherTok}[1]{\textcolor[rgb]{0.00,0.23,0.31}{#1}}
\newcommand{\PreprocessorTok}[1]{\textcolor[rgb]{0.68,0.00,0.00}{#1}}
\newcommand{\RegionMarkerTok}[1]{\textcolor[rgb]{0.00,0.23,0.31}{#1}}
\newcommand{\SpecialCharTok}[1]{\textcolor[rgb]{0.37,0.37,0.37}{#1}}
\newcommand{\SpecialStringTok}[1]{\textcolor[rgb]{0.13,0.47,0.30}{#1}}
\newcommand{\StringTok}[1]{\textcolor[rgb]{0.13,0.47,0.30}{#1}}
\newcommand{\VariableTok}[1]{\textcolor[rgb]{0.07,0.07,0.07}{#1}}
\newcommand{\VerbatimStringTok}[1]{\textcolor[rgb]{0.13,0.47,0.30}{#1}}
\newcommand{\WarningTok}[1]{\textcolor[rgb]{0.37,0.37,0.37}{\textit{#1}}}

\providecommand{\tightlist}{%
  \setlength{\itemsep}{0pt}\setlength{\parskip}{0pt}}\usepackage{longtable,booktabs,array}
\usepackage{calc} % for calculating minipage widths
% Correct order of tables after \paragraph or \subparagraph
\usepackage{etoolbox}
\makeatletter
\patchcmd\longtable{\par}{\if@noskipsec\mbox{}\fi\par}{}{}
\makeatother
% Allow footnotes in longtable head/foot
\IfFileExists{footnotehyper.sty}{\usepackage{footnotehyper}}{\usepackage{footnote}}
\makesavenoteenv{longtable}
\usepackage{graphicx}
\makeatletter
\def\maxwidth{\ifdim\Gin@nat@width>\linewidth\linewidth\else\Gin@nat@width\fi}
\def\maxheight{\ifdim\Gin@nat@height>\textheight\textheight\else\Gin@nat@height\fi}
\makeatother
% Scale images if necessary, so that they will not overflow the page
% margins by default, and it is still possible to overwrite the defaults
% using explicit options in \includegraphics[width, height, ...]{}
\setkeys{Gin}{width=\maxwidth,height=\maxheight,keepaspectratio}
% Set default figure placement to htbp
\makeatletter
\def\fps@figure{htbp}
\makeatother

% Use adjustwidth environment to exceed column width (see example table in text)
\usepackage{changepage}

% marvosym package for additional characters
\usepackage{marvosym}

% cite package, to clean up citations in the main text. Do not remove.
% Using natbib instead
% \usepackage{cite}

% Use nameref to cite supporting information files (see Supporting Information section for more info)
\usepackage{nameref,hyperref}

% line numbers
\usepackage[right]{lineno}

% ligatures disabled
\usepackage{microtype}
\DisableLigatures[f]{encoding = *, family = * }

% create "+" rule type for thick vertical lines
\newcolumntype{+}{!{\vrule width 2pt}}

% create \thickcline for thick horizontal lines of variable length
\newlength\savedwidth
\newcommand\thickcline[1]{%
  \noalign{\global\savedwidth\arrayrulewidth\global\arrayrulewidth 2pt}%
  \cline{#1}%
  \noalign{\vskip\arrayrulewidth}%
  \noalign{\global\arrayrulewidth\savedwidth}%
}

% \thickhline command for thick horizontal lines that span the table
\newcommand\thickhline{\noalign{\global\savedwidth\arrayrulewidth\global\arrayrulewidth 2pt}%
\hline
\noalign{\global\arrayrulewidth\savedwidth}}

% Text layout
\raggedright
\setlength{\parindent}{0.5cm}
\textwidth 5.25in 
\textheight 8.75in

% Bold the 'Figure #' in the caption and separate it from the title/caption with a period
% Captions will be left justified
\usepackage[aboveskip=1pt,labelfont=bf,labelsep=period,justification=raggedright,singlelinecheck=off]{caption}
\renewcommand{\figurename}{Fig}

% Remove brackets from numbering in List of References
\makeatletter
\renewcommand{\@biblabel}[1]{\quad#1.}
\makeatother

% Header and Footer with logo
\usepackage{lastpage,fancyhdr}
\usepackage{epstopdf}
%\pagestyle{myheadings}
\pagestyle{fancy}
\fancyhf{}
%\setlength{\headheight}{27.023pt}
%\lhead{\includegraphics[width=2.0in]{PLOS-submission.eps}}
\rfoot{\thepage/\pageref{LastPage}}
\renewcommand{\headrulewidth}{0pt}
\renewcommand{\footrule}{\hrule height 2pt \vspace{2mm}}
\fancyheadoffset[L]{2.25in}
\fancyfootoffset[L]{2.25in}
\lfoot{\today}
% Remove comment for double spacing
% \usepackage{setspace}
% \doublespacing
\makeatletter
\makeatother
\makeatletter
\makeatother
\makeatletter
\@ifpackageloaded{caption}{}{\usepackage{caption}}
\AtBeginDocument{%
\ifdefined\contentsname
  \renewcommand*\contentsname{Table of contents}
\else
  \newcommand\contentsname{Table of contents}
\fi
\ifdefined\listfigurename
  \renewcommand*\listfigurename{List of Figures}
\else
  \newcommand\listfigurename{List of Figures}
\fi
\ifdefined\listtablename
  \renewcommand*\listtablename{List of Tables}
\else
  \newcommand\listtablename{List of Tables}
\fi
\ifdefined\figurename
  \renewcommand*\figurename{Figure}
\else
  \newcommand\figurename{Figure}
\fi
\ifdefined\tablename
  \renewcommand*\tablename{Table}
\else
  \newcommand\tablename{Table}
\fi
}
\@ifpackageloaded{float}{}{\usepackage{float}}
\floatstyle{ruled}
\@ifundefined{c@chapter}{\newfloat{codelisting}{h}{lop}}{\newfloat{codelisting}{h}{lop}[chapter]}
\floatname{codelisting}{Listing}
\newcommand*\listoflistings{\listof{codelisting}{List of Listings}}
\makeatother
\makeatletter
\@ifpackageloaded{caption}{}{\usepackage{caption}}
\@ifpackageloaded{subcaption}{}{\usepackage{subcaption}}
\makeatother
\makeatletter
\makeatother
\ifLuaTeX
  \usepackage{selnolig}  % disable illegal ligatures
\fi
\usepackage[numbers,square,comma]{natbib}
\bibliographystyle{plos2015}
\IfFileExists{bookmark.sty}{\usepackage{bookmark}}{\usepackage{hyperref}}
\IfFileExists{xurl.sty}{\usepackage{xurl}}{} % add URL line breaks if available
\urlstyle{same} % disable monospaced font for URLs
\hypersetup{
  pdftitle={Longitudinal Analysis Manuscript: Working Draft},
  pdfauthor={Samuel W. Hawes, PhD.; Andrew K. Littlefield, PhD.; Daniel A. Lopez; Kenneth J. Sher; Wesley K. Thompson; Additional Co-authors1; Additional Co-authors2},
  hidelinks,
  pdfcreator={LaTeX via pandoc}}



\begin{document}
\vspace*{0.2in}

% Title must be 250 characters or less.
\begin{flushleft}
{\Large
\textbf\newline{Longitudinal Analysis Manuscript: Working
Draft} % Please use "sentence case" for title and headings (capitalize only the first word in a title (or heading), the first word in a subtitle (or subheading), and any proper nouns).
}
\newline
\\
% Insert author names, affiliations and corresponding author email (do not include titles, positions, or degrees).
Samuel W. Hawes, PhD.\textsuperscript{1*}, Andrew K. Littlefield,
PhD.\textsuperscript{2}, Daniel A. Lopez\textsuperscript{3}, Kenneth J.
Sher\textsuperscript{4}, Wesley K.
Thompson\textsuperscript{5}, Additional
Co-authors1\textsuperscript{1}, Additional
Co-authors2\textsuperscript{1}
\\
\bigskip
\textbf{1} Center for Children and Families, Florida International
University, Miami, FL, United States, \\ \textbf{2} Psychological
Sciences, Texas Tech University, Lubbock, TX, United
States, \\ \textbf{3} Department of Public Health, University of
Rochester Medical Center, Rochester, NY, United
States, \\ \textbf{4} Psychological Sciences, University of
Missouri, Columbia, MO, United States, \\ \textbf{5} Population
Neuroscience and Genetics Center, Laureate Institute for Brain
Research, Tulsa, OK, United States, 
\bigskip

% Insert additional author notes using the symbols described below. Insert symbol callouts after author names as necessary.
% 
% Remove or comment out the author notes below if they aren't used.
%
% Primary Equal Contribution Note
\Yinyang These authors contributed equally to this work.

% Additional Equal Contribution Note
% Also use this double-dagger symbol for special authorship notes, such as senior authorship.
%\ddag These authors also contributed equally to this work.

% Current address notes
\textcurrency Current Address: Dept/Program/Center, Institution Name, City, State, Country % change symbol to "\textcurrency a" if more than one current address note
% \textcurrency b Insert second current address 
% \textcurrency c Insert third current address

% Deceased author note
\dag Deceased

% Group/Consortium Author Note
\textpilcrow Biostatistics Working Group membership list can be found in
the Acknowledgments sections

% Use the asterisk to denote corresponding authorship and provide email address in note below.
* shawes@fiu.edu

\end{flushleft}

\section*{Abstract}
The Adolescent Brain Cognitive Development (ABCD) Study presents a
unique opportunity for researchers to investigate developmental
processes in a large, diverse cohort of children and adolescents. Given
the complex nature of the longitudinal data collected in the ABCD Study,
researcher are likely to encounter a myriad of methodological and
analytic considerations and concerns. This review provides a
comprehensive examination of key issues and techniques related to
longitudinal data analysis, specifically focusing on the ABCD Study. The
text discusses model assumptions, common violations (e.g., independent
and identically distributed residuals, \emph{heterogeneous}
\emph{variability}) and their implications for valid inference. The
importance of appropriately modeling covariance structures,
understandings trade-offs between model fit and parsimony, and
challenges related to sample size, attrition, missing data are
highlighted. Consideration is given to the importance of selecting
appropriate statistical models to account for correlations in repeated
measurements and the assumptions underlying these models. The review
also differentiates between linear and non-linear models in the context
of continuous and discrete data, emphasizing various distributional
assumptions and the necessity of choosing appropriate models and
statistical methods. By addressing these complexities, the review seeks
to equip researchers with the necessary knowledge and tools to make
informed decisions as they navigate effectively analyzing and
interpreting data available in the ABCD Study.

\section*{Author summary}
The Adolescent Brain Cognitive Development (ABCD) Study offers valuable
insights into adolescent brain development. Researchers working with
this data face methodological and analytic concerns, including sample
size, attrition, missing data, and measurement. This review addresses
complexities in analyzing longitudinal data, with a focus on statistical
models accounting for correlations in repeated measurements and their
assumptions. We emphasize the importance of choosing appropriate models
and analytic functions based on the data and the specific scientific
question to be addressed. This concise resource aims to help researchers
navigate longitudinal data analysis, make informed methodological
decisions, and enhance the accuracy and reliability of their findings
when working with data from the ABCD Study.

\linenumbers\hypertarget{introduction}{%
\section{Introduction}\label{introduction}}

\label{sec:headings} The Adolescent Brain Cognitive Development (ABCD)
Study® is the largest long-term investigation of neurodevelopment and
child health in the United States. Conceived and initiated by the
National Institutes of Health (NIH), this landmark prospective
longitudinal study aims to transform our understanding of the genetic
and environmental influences on brain development and their roles in
behavioral and health outcomes in adolescents \citep{volkow2018}. At its
heart, the study is designed to chart the course of human development
across multiple, interacting domains from late childhood to early
adulthood and to identify factors that lead to both positive and
negative developmental outcomes. Central to achieving these goals is the
ABCD Study's® commitment to an open science framework designed to
facilitate access to and sharing of scientific knowledge by espousing
practices that increase openness, integrity, and reproducibility of
scientific research (e.g., public data releases). In this sense, the
ABCD Study® is a collaboration with the larger research community, with
the rich longitudinal nature of the ABCD Study dataset allowing
researchers to perform a variety of analyses of both methodological and
substantive interest. Together, this presents a unique opportunity to
significantly advance our understanding of how a multitude of
biopsychosocial processes emerge and unfold across critical periods of
development.

{[}section still be developed\ldots{]}

\hypertarget{the-abcd-study-data}{%
\subsection{The ABCD Study® Data}\label{the-abcd-study-data}}

Participants enrolled in the ABCD Study include a large cohort of youth
(n=11880) aged 9-10 years at baseline and their parents/guardians. The
study sample was recruited from household populations in defined
catchment areas for each of the 21 study sites across the United States
(information regarding funding agencies, recruitment sites,
investigators, and project organization can be obtained at the ABCD
Study website). The ABCD Study is collecting longitudinal data on a rich
variety of outcomes that will enable the construction of
realistically-complex etiological models by incorporating factors from
many domains simultaneously. Each new wave of data collection provides
the building blocks for conducting probing longitudinal analyses that
allow us to characterize normative development, identify variables that
presage deviations from prototypic development, and assess a range of
outcomes associated with variables of interest. This data includes a
neurocognitive battery \citep{luciana2018a, thompson2019}, mental and
physical health assessments \citep{barch2018}, measures of culture and
environment \citep{zucker2018}, substance use {[}add citation{]},
biospecimens \citep{uban2018}, structural and functional brain imaging
\citep{casey2018, hagler2019}, geolocation-based environmental exposure
data, wearables, and mobile technology \citep{bagot2018}, and whole
genome genotyping \citep{loughnan2020}. Many of these measures are
collected at in-person annual visits, with brain imaging collected at
baseline and every other year going forward. A limited number of
assessments are collected in semi-annual telephone interviews between
in-person visits. Data are publicly released on an annual basis through
the NIMH Data Archive. By necessity, the study's earliest data releases
were cross-sectional (i.e., the baseline data), however, the most recent
public data release (NDA Release 4.0) contains data collected across
three annual assessments, including two imaging assessments (baseline
and year 2 follow-up visits).

\hypertarget{organization-of-current-manuscript}{%
\subsection{Organization of current
manuscript}\label{organization-of-current-manuscript}}

The rich longitudinal nature of the ABCD Study dataset will allow
researchers to perform analyses of both methodological and substantive
interest. This report describes methods for longitudinal analyses of
ABCD Study data that can address its fundamental scientific aims, as
well as challenges inherent in a large population-based long-term study
of adolescents. The manuscript is organized as follows:

{[}section still be developed\ldots{]}

\hypertarget{part-i-longitudinal-research-basic-concepts-and-considerations}{%
\section{Part I: Longitudinal Research: Basic Concepts and
Considerations}\label{part-i-longitudinal-research-basic-concepts-and-considerations}}

\label{sec:headings} There are several important concepts to consider
when conducting longitudinal analyses in a developmental context. These
include different ways of thinking about developmental course, whether
certain periods of development are relatively sensitive or insensitive
to various types of insults or stressors, whether some time periods or
situations inhibit the expression of individual differences due to
extreme environmental pressures, and whether the same behavior
manifested at different times represent the same phenomenon or different
ones. Further, in the case of developmentally focused longitudinal
research, each new measurement occasion not only provides a more
extended portrait of the child's life course (and not just characterize
growth during this period but also assesses the durability/chronicity of
prior effects/consequences) but also brings with it greater
methodological opportunities to exploit the statistical properties of
longitudinal data in the furtherance of critical scientific questions.
That is, we can ask more nuanced questions and make stronger inferences
as our number of time-ordered observations grows, assuming we have
assessed the ``right'' variables and the timings of our observations
comport with the temporal dynamics of the mechanisms of interest.
Appreciation of these and other issues can help to guide the analysis
and interpretation of data and aid translation to clinical and public
health applications.

\textbf{Vulnerable periods}. Development normatively progresses from
less mature to more mature levels of functioning. However, unique epochs
and experiences can alter the course of this idealized form of
development. Consider research that shows cannabis use during
adolescence is associated with later psychosis to a greater degree than
cannabis use initiated later in development {[}add citation{]}; or,
similarly, experimental research on rodents that shows rodent brains to
be especially sensitive to the neurotoxic effects of alcohol on brain
structure and learning early in development (corresponding to early
adolescence in humans){[}add citation{]}. These examples highlight the
importance of considering the role of vulnerable periods -- temporal
windows of rapid brain development or remodeling during which the
effects of environmental stimuli (e.g.~cannabis exposure) on the
developing brain may be particularly pronounced-- when trying to
establish an accurate understanding of the association between exposures
and outcomes.

\textbf{Developmental disturbances}. Whereas vulnerable periods heighten
neurobiological susceptibility to environmental influences, at other
times environmental pressures will tend to suppress stability and
disrupt the orderly stochastic process of normative development (e.g.,
xxx-xxx). This situation reflects a developmental disturbance in that
the normal course of development is ``disturbed'' for some time by some
time-limited process. In such cases, we might find that prediction of
behavior in the period of the disturbance is reduced and/or, similarly,
the behavior exhibited during the disturbance might have less predictive
power with respect to distal outcomes compared to the behavior exhibited
before and following the disrupted period. That is, once the
environmental stimuli are removed (or the individual is removed from the
environment), individual differences are again more freely expressed and
the autoregressive effects increase to levels similar to those before
entering the environment.

\textbf{Developmental snares and cascade effects}. Normative development
can also be upended by experiences (e.g., drug use) that, through
various mechanisms, disrupt the normal flow of development wherein each
stage establishes a platform for the next. For instance, substance use
could lead to association with deviant peers, precluding opportunities
for learning various adaptive skills and prosocial behaviors, in effect,
creating a ``snare'' that retards psychosocial development. Relatedly,
the consequences of these types of events can cascade (e.g., school
dropout, involvement in the criminal justice system) so that the effects
of the snare are amplified. Although conceptually distinct from
vulnerable periods, both of these types of developmental considerations
highlight the importance of viewing behavior in the context of
development and the importance of attempting to determine how various
developmental pathways unfold.

\textbf{Distinguishing developmental change from experience effects}.
One can often observe systematic changes over time in a variable of
interest and assume this change is attributable to development. To this
point, cognitive abilities (e.g, verbal ability, problem-solving)
normatively grow earlier in development and often decline in late life
(e.g., memory, speed of processing). However, the observed patterns of
growth and decline often differ between cross-sectional vs.~longitudinal
effects \citep{salthouse2014} where subjects gain increasing experience
with the assessment with each successive measurement occasion. Such
experience effects on cognitive functioning have been demonstrated in
adolescent longitudinal samples similar to ABCD \citep{sullivan2017} and
highlight the need to consider these effects and address them
analytically. In the case of performance-based measures {[}e.g., matrix
reasoning related to neurocognitive functioning; see
\citet{salthouse2014}{]}, this can be due to ``learning'' the task from
previous test administrations (e.g., someone taking the test a second
time performs better than they did the first time simply as a function
of having taken it before). Even in the case of non-performance-based
measures (e.g., levels of depression), where one cannot easily make the
argument that one has acquired some task-specific skill through
learning, it has been observed that respondents tend to endorse lower
levels on subsequent assessments
\citetext{\citealp[e.g.,][]{beck1961}; \citealp[see][]{french2010}} and
this phenomenon has been well documented in research on structured
diagnostic interviews \citep{robins1985}. While it is typically assumed
that individuals are rescinding or telling us less information on
follow-up interviews, there is reason to suspect that in some cases the
initial assessment may be artefactually elevated
\citep[see][]{shrout2018a}. Some designs (specifically, accelerated
longitudinal designs) are especially well suited for discovering these
effects and modeling them. While ABCD was not designed as an accelerated
longitudinal design, the variability in age at the time of baseline
recruitment (9 years, 0 months to 10 years, 11 months) allows some
measures, collected every year, to be conceptualized as an accelerated
longitudinal design. Moreover, it is possible that in later waves,
patterns of longitudinal missing data will allow some analyses to assess
the confounded effects of age and the number of prior assessments.
However, ABCD is fundamentally a single-cohort, longitudinal design,
where a number of prior assessments and age are highly confounded, and
for, perhaps, most analyses, the possible influence of experience
effects needs to be kept in mind.

\hypertarget{part-ii-longitudinal-data-interpretation-issues-pitfalls-assumption}{%
\section{Part II Longitudinal Data: Interpretation / Issues / Pitfalls
\&
Assumption}\label{part-ii-longitudinal-data-interpretation-issues-pitfalls-assumption}}

\label{sec:headings} \textbf{Defining Features of Longitudinal Data
Analysis.} The hallmark characteristic of longitudinal data analysis is
its application to repeated assessments of the same assessment targets
(e.g., individuals, families) across time. While the primary reason for
collecting longitudinal data is in pursuit of addressing scientific
questions, from a methodological perspective, having multiple
observations over time allows researchers to identify potentially
problematic observations when highly improbable longitudinal patterns
are observed. That is, we can ask more nuanced questions and make
stronger inferences as our number of time-ordered observations grows
assuming we have assessed the ``right'' variables and the timings of our
observations comport with the temporal dynamics of the mechanisms of
interest .

\hypertarget{modeling-data-across-two-time-points-versus-three-or-more-time-points.}{%
\subsection{Modeling Data Across Two Time Points versus Three or More
Time
Points.}\label{modeling-data-across-two-time-points-versus-three-or-more-time-points.}}

Although the clear leap to the realm of longitudinal data involves going
from one assessment to two or more assessments, there are also notable
distinctions in designs based on two-assessment points versus three or
more measurement occasions. Just as cross-sectional data can be
informative in some situations, two waves of data can be beneficial in
contexts such as when experimental manipulation is involved (e.g.,
pre/post tests), or if the central goal is prediction (e.g., trying to
predict scores on Variable A at time T as a function of prior scores on
Variable A and Variable B at time T-1). At the same time, data based on
two assessments are inherently limited on multiple fronts. As
\citep{rogosa1982} noted approximately forty years ago, ``Two waves of
data are better than one, but maybe not much better''. These sentiments
are reflected in more contemporary recommendations regarding
best-practice guidelines for prospective data, which increasingly
emphasize the benefits of additional measurement occasions for model
identification and accurate parameter estimation. It is also consistent
with research recommending that developmental studies include three or
more assessment points, given it is impossible for data based on
two-time points to determine the shape of development (since linear,
straight line change is the only possible form, given two assessments;
see \citep{duncan2009}). Research designs that include three or more
time points allow for increasingly nuanced analyses that more adequately
tease apart sources of variation and covariation among the repeated
assessments \citep{king2018}-- a key aspect of inferential research. To
illustrate, developmental theories are typically interested in
understanding patterns of within-individual change over time (discussed
in further detail, below); however, two data points provide meager
information on change at the person level. This point is further
underscored in a recent review of statistical models commonly touted as
distinguishing within-individual vs between-individual sources of
variance in which the study authors concluded ``\ldots{} researchers are
limited when attempting to differentiate these sources of variation in
psychological phenomenon when using two waves of data'' and perhaps more
concerning, ``\ldots the models discussed here do not offer a feasible
way to overcome these inherent limitations''
\citet{littlefield20210603}. It is important to note, however, that
despite the current focus on two-wave designs versus three or more
assessment waves, garnering three assessment points is not a panacea for
longitudinal modeling. Indeed, several contemporary longitudinal models
designed to isolate within-individual variability {[}e.g., the Latent
Curve Model with Structured Residuals; \citet{curran2014a}{]} require at
least four assessments to parameterize fully and, more generally,
increasingly accurate parameter estimates are obtained as more
assessment occasions are used \citep{duncan2009}.

\hypertarget{types-of-stability-and-change}{%
\subsection{Types of stability and
change}\label{types-of-stability-and-change}}

If one were to try to sum up what development in a living organism is
exactly, one could plausibly argue it's the characterization of
stability and change as the organism traverses the life course. There
are a few different ways to think of stability (and change). Consider we
measure the height of all youth in a 6th-grade class, once in the fall
at the beginning of the school year and once again in the spring at the
end of the school year. A common first step may be to compare the
class's average height values obtained at these two different
measurement occasions. This comparison of the average scores for the
same group of individuals at multiple time points is referred to as
``mean-level'' stability as it provides information about continuity and
change in the group level of an outcome of interest (e.g., height) over
time. Another type of stability involves calculating the correlation
between the values obtained at different time points (e.g., `height in
the fall' with `height in the spring'). This type of ``rank-order''
stability evaluates between-individual change by focusing on the degree
to which individuals retain their relative placement in a group across
time. Consider, someone who is the shortest person in their class in 6th
grade may grow considerably over the school year (i.e., exhibit mean
level change), but remain the shortest person among their classmates.
That is, the individual is manifesting a type of rank-order stability.
Both types of stability and change are important. Mean-level change in
certain traits might help to explain why, in general, individuals are
particularly vulnerable to social influences at some ages more than
others; rank order change might help to quantify the extent to which
certain characteristics of the individual are more trait-like. For
example, in some areas of development, there is considerable mean-level
change that occurs over time (e.g., changes in Big 5 personality
traits), but relatively high rank-order stability. Despite the useful
information afforded by examining mean-level and rank-order change,
these approaches are limited in that they provide little information
about patterns of ``within-individual'' change and, in turn, can result
in fundamental misinterpretations about substantial or meaningful
changes in an outcome of interest.

There is growing recognition that statistical models commonly applied to
longitudinal data often fail to comport with the developmental theory
they are being used to assess (e.g., Curran, Lee, Howard, Lane, \&
MacCallum, 2012; Hoffman, 2015; Littlefield et al., 2021. Specifically,
developmental studies typically involve the use of prospective data to
inform theories that are concerned with clear within-person (i.e.,
intraindividual) processes (e.g., how phenotypes change or remain stable
within individuals over time) \citep[e.g., see][]{curran2011}. Despite
this, methods generally unsuited for disaggregating between- and
within-person effects (e.g., cross-lagged panel models {[}CLPM{]})
remain common within various extant literatures. As a result, experts
increasingly caution about the need to xxxxxxxx {[}add citation{]}.
Fortunately, there exists a range of models that have been proposed to
tease apart between- and within-person sources of variance across time
\citep[see][]{littlefield20210603, orth2021}. Most of these contemporary
alternatives incorporate time-specific latent variables to capture
between-person sources of variance and model within-person deviations
around an individual's mean (or trait) level across time
\citetext{\citealp[e.g.,
RI-CLPM,][]{hamaker2015}; \citealp[LCM-SR,][]{curran2014a}}. It is
important to note however that these models require multiple assessments
waves (e.g., four or more to fully specify the LCM-SR), additional
expertise to overcome issues with model convergence, and appreciation of
modeling assumptions when attempting to adjudicate among potential
models in each research context \citep[see][for further
discussion]{littlefield20210603}.

\hypertarget{model-assumptions}{%
\subsection{Model Assumptions}\label{model-assumptions}}

Many statistical models assume certain characteristics about the data to
which they are being applied. As an example, common assumptions of
parametric statistical models include normality, linearity, and equality
of variances. These assumptions must be carefully considered before
conducting analysis so that valid inferences can be made from the data;
that is, violation of a model's assumptions can substantively alter the
interpretation of results. Similarly, statistical models employed in the
analyses of longitudinal data often entail a range of assumptions that
must be closely inspected. One central issue for repeated measurements
on an individual is how to account for the correlated nature of the
data; another common feature of longitudinal data is heterogeneous
variability; that is, the variance of the response changes over the
duration of the study. Traditional techniques, such as a standard
regression or ANOVA model, assume residuals are independent and thus are
inappropriate for designs that assess (for example) the same individuals
across time. That is, given the residuals are no longer independent, the
standard errors from the models are biased and can produce misleading
inferential results. Although there are formal tests of independence for
time series data (e.g., the Durbin-Watson statistic; Durbin \& Watson,
1950), more commonly independence is assumed to be violated in study
designs with repeated assessments. Therefore, an initial question to be
addressed by a researcher analyzing prospective data is how to best
model the covariance structure of said data.

\hypertarget{covariance-structures}{%
\subsection{Covariance Structures}\label{covariance-structures}}

Statistical models for longitudinal data include two main components to
account for assumptions that are commonly violated when working with
repeated measures data: a model for the covariance among repeated
measures (both the correlations among pairs of repeated measures on an
individual and the variability of the responses on different occasions),
coupled with a model for the mean response and its dependence on
covariates (eg, treatment group in the context of clinical trials). This
allows for the specification of a range of so-called covariance
structures, each with its own set of tradeoffs between model fit and
parsimony \citep[e.g., see][]{kincaid2005}.

\hypertarget{accounting-for-correlated-data}{%
\subsection{Accounting for Correlated
Data}\label{accounting-for-correlated-data}}

As an example, one alternative structure that attempts to handle the
reality that correlations between repeated assessments tend to diminish
across time is the autoregressive design. As the name implies, the
structure assumes a subsequent measurement occasion (e.g., assessment at
Wave 2) is regressed onto (that is, is predicted by) a prior measurement
occasion (e.g., assessment at Wave 1). The most common type of
autoregressive design is the AR(1), where assessments at time T + 1 are
regressed on assessments at Time T. Identical to compound symmetry, this
model assumes the variances are homogenous across time. Diverting from
compound symmetry, this model assumes the correlations between repeated
assessments decline exponentially across time rather than remaining
constant. For example, per the AR(1) structure, if the correlation
between Time 1 and Time 2 data is thought to be .5, then the correlation
between Time 1 and Time 3 data would be assumed to be .5\emph{.5 = .25,
and the correlation between Time 1 and Time 4 data would be assumed to
be .5}.5*.5 = .125. As with compound symmetry, the basic AR(1) model is
parsimonious in that it only requires two parameters (the variance of
the assessments and the autoregressive coefficient). Notably, the
assumption of constant autoregressive relations between assessments is
often relaxed in commonly employed designs that use autoregressive
modeling (e.g., cross-lagged panel models {[}CLPM{]}). These designs
still typically assume an AR(1) process (e.g., it is sufficient to
regress the Time 3 assessment onto the Time 2 assessment and is not
necessary to also regress the Time 3 assessment onto the Time 1
assessment, which would result in an AR(2) process). However, the
magnitude of these relations is often allowed to differ across different
AR(1) pairs of assessment (e.g., the relation between Time 1 and Time 2
can be different from the relation between Time 2 and Time 3). These
more commonly employed models also often relax the assumption of equal
variances of the repeated assessments. Although the AR(1) structure may
involve a more realistic set of assumptions compared to compound
symmetry, in that the AR(1) model allows for diminishing correlations
across time, the basic AR(1) model, as well as autoregressive models
more generally, can also suffer from several limitations in contexts
that are common in prospective designs. In particular, recent work
demonstrates that if a construct being assessed prospectively across
time is trait-like in nature, then autoregressive relations fail to
adequately account for this trait-like structure, with the downstream
consequence that estimates derived from models based on AR structures
(such as the CLPM) can be misleading and fail to adequately demarcate
between- vs.~within-person sources of variance \citep{hamaker2015}.

\hypertarget{linear-vs-non-linear-models}{%
\subsection{Linear vs non-linear
models}\label{linear-vs-non-linear-models}}

Identification of optimal statistical models and appropriate
mathematical functions requires an understanding of the type of data
being used. Repeated assessments can be based on either continuous or
discrete measures. Examples of discrete measures include repeated
assessments of binary variables (e.g., past 12-month alcohol use
disorder status measured across ten years), ordinal variables (e.g., a
single item measuring the level of agreement to a statement on a
three-point scale including the categories of ``disagree'', ``neutral'',
and ``agree'' in an ecological momentary assessment study that involves
multiple daily assessments), and count variables (e.g., number of
cigarettes smoked per day across a daily diary study). In many ways, the
distributional assumptions of indicators used in longitudinal designs
mirror the decision points and considerations when delineating across
different types of discrete outcome variables, a topic that spans entire
textbooks \citep[e.g., see][]{lenz2016}. For example, the Mplus manual
\citep{muthen2017} includes examples of a) censored and
censored-inflated models, b) linear growth models for binary or ordinal
variables, c) linear growth models for a count outcome assuming a
Poisson model, d) linear growth models for a count outcome assuming a
zero-inflated Poisson model and e) discrete- and continuous-time
survival analysis for a binary outcome. Beyond these highlighted
examples, other distributions (e.g., negative binomial) can be assumed
for the indicators when modeling longitudinal data. These models can
account for issues that can occur when working with discrete outcomes,
including overdispersion (when the variance is higher than would be
expected based on a given distribution) and zero-inflation {[}when more
zeros occur than is expected based on a given distribution; see
\citet{lenz2016}{]}. Models involving zero-inflation parameters are
referred to as two-part models, given one part of the model predicts the
zero-inflation whereas the other part of the model predicts outcomes
consistent with a given distribution {[}e.g., Poisson distribution; see
\citet{farewell2017}, for a review of two-part models for longitudinal
data{]}. Although there exist several alternative models for discrete
indicators, some more recent models that have been proposed for
prospective data are only feasible in cases where indicators are assumed
to be continuous rather than discrete {[}e.g., LCM-SR;
\citet{curran2014a}{]}. Given the sheer breadth of issues relevant to
determining better models for discrete outcomes, it is not uncommon for
texts on longitudinal data analysis to only cover models and approaches
that assume continuous indicators \citep[e.g.,][]{little2013}. However,
some textbooks on categorical data analysis provide more detailed
coverage of the myriad issues and modeling choices to consider when
working with discrete outcomes {[}e.g., \citet{lenz2016}, Chapter 11 for
matched pair/two-assessment designs; Chapter 12 for marginal and
transitional models for repeated designs, such as generalized estimating
equations, and Chapter 13 for random effects models for discrete
outcomes{]}.

\hypertarget{missing-dataattrition}{%
\subsection{Missing Data/Attrition}\label{missing-dataattrition}}

As recently reviewed by Littlefield (in press), investigators of
prospective data are confronted with study attrition (i.e., participants
may not provide data at a given wave of assessment) and thus approaches
are needed to confront the issue of missing data. Three models of
missingness are typically considered in the literature
\citep[see][]{little1989}. These three models are data: a) missing
completely at random (MCAR), b) missing at random (MAR), and c) missing
not at random (MNAR). Data that are MCAR means missing data is a random
sample of all the types of participants (e.g., males) in a given
dataset. MAR suggests conditionally missing at random
\citep[see][]{graham2009}. That is, MAR implies missingness is
completely random (i.e., does not hinge on some unmeasured variables)
once missingness has been adjusted by all available variables in a
dataset (e.g., biological sex). Data that are MNAR are missing as a
function of unobserved variables. \citet{graham2009} provides an
excellent and easy-to-digest overview of further details involving
missing data considerations.

Multiple approaches have been posited to handle missing data. Before the
advent of more contemporary approaches, common methods included several
ad hoc procedures. These include eliminating the data of participants
with missing data (e.g., listwise or pairwise deletion) or using mean
imputation (i.e., replacing the missing value with the mean score of the
sample that did participate). However, these methods are not recommended
because they can contribute to biased parameter estimates and research
conclusions \citep[see][]{graham2009}. More specifically, the last
observation carried forward (LOCF) is a common approach to imputing
missing data. LOCF replaces a participant's missing values after dropout
with the last available measurement \citep{molnar2008}. This approach
assumes stability (i.e., a given participant's score is not anticipated
to increase or decline after study dropout) and that the data are MCA R.
However, as described by \citet{molnar2008}, it is common for treatment
groups to show higher attrition compared to control groups in studies of
dementia drugs. Given that dementia worsens over time, using LOCF biases
the results in favor of the treatment group \citep[see][for more
details]{molnar2008}.

More modern approaches, such as using maximum likelihood or multiple
imputation to estimate missing data, are thought to avoid some of the
biases of older approaches \citep[see][]{enders2010, graham2009}.
\citet{graham2009} noted several ``myths'' regarding missing data. For
example, Graham notes many assume the data must be minimally MAR to
permit estimating procedures (such as maximum likelihood or multiple
imputation) compared to other, more traditional approaches (e.g., using
only complete case data). Violations of MAR impact both traditional and
more modern data estimation procedures, though as noted by Graham,
violations of MAR tend to have a greater effect on older methods. Graham
thus suggests that estimating missing data is a better approach compared
to the older procedures in most circumstances, regardless of the model
of missingness {[}i.e., MCAR, MAR, MNAR; see \citet{graham2009}{]}.

Attrition from a longitudinal panel study such as ABCD is inevitable and
represents a threat to the validity of longitudinal analyses and
cross-sectional analyses conducted at later time points, especially
since attrition can only be expected to grow over time. While, to date,
attrition in ABCD has been minimal (some cite here), it remains an
important focus for longitudinal analysis and its significance is likely
to only grow as the cohort ages. Ideally, one tries to minimize
attrition through good retention practices from the outset via
strategies designed to maintain engagement in the project
\citep{cotter2005, hill2016, watson2018}. However, even the
best-executed studies need to anticipate growing attrition over the
length of the study and implement analytic strategies designed to
provide the most valid inferences. Perhaps the most key concern when
dealing with data that is missing due to attrition is determining the
degree of bias in retained variables that is a consequence of attrition.
Assuming that the data are not missing completely at random, attention
to the nature of the missingness and employing techniques designed to
mitigate attrition-related biases need to be considered in all
longitudinal analyses. Several different approaches can be considered
and employed depending upon the nature of the intended analyses, the
degree of missingness, and data available to help estimate missing and
unobserved values.

\hypertarget{quantifying-effect-sizes-longitudinally}{%
\subsection{Quantifying effect sizes
longitudinally}\label{quantifying-effect-sizes-longitudinally}}

Given longitudinal data involve different sources of variance,
quantifying effect sizes longitudinally is a more difficult task
compared to deriving such estimates from cross-sectional data. Effect
size can be defined as, ``a population parameter (estimated in a sample)
encapsulating the practical or clinical importance of a phenomenon under
study.'' \citep{kraemer2014}. Common effect size metrics include r
(i.e., the standardized covariance, or correlation, between two
variables) and Cohen's d \citep{cohen1988}. Adjustments to common effect
size calculations, such as Cohen's d, are required even when only two
time points are considered \citep[e.g., see][]{morris2002}.
\citet{wang2019a} note there are multiple approaches to obtaining
standardized within-person effects, and that commonly suggested
approaches (e.g., global standardization) can be problematic
\citep[see][for more details]{wang2019a}. Thus, obtaining effect size
metrics based on standardized estimates that are relatively simple in
cross-sectional data (such as r) becomes more complex in the context of
prospective data. \citet{feingold2009} noted that equations for effects
sizes used in studies involving growth modeling analysis (e.g., latent
growth curve modeling) were not mathematically equivalent, and the
effect sizes were not in the same metric as effect sizes from
traditional analysis \citep[see][for more details]{feingold2009}. Given
this issue, there have been various proposals for adjusting effect size
measures in repeated assessments. \citet{feingold2019} reviews the
approach for effect size metrics for analyses based on growth modeling,
including when considering linear and non-linear (i.e., quadratic)
growth factors. \citet{morris2002} review various equations for effect
size calculations relevant to when combining estimates in meta-analysis
with repeated measures and independent-groups designs. Other approaches
to quantifying effect sizes longitudinally may be based on standardized
estimates from models that more optimally disentangle between- and
within-person sources of variance (as reviewed above). As an example,
within a RI-CLPM framework, standardized estimates between random
intercepts (i.e., the correlation between two random intercepts for two
different constructs assessed repeatedly) could be used to index the
between-person relation, whereas standardized estimates among the
structured residuals could be used as informing the effect sizes of
within-person relations.

\hypertarget{general-guidelines-for-this-quarto-template}{%
\section{General guidelines for this Quarto
template}\label{general-guidelines-for-this-quarto-template}}

This template shows how to use PLOS template from
\url{https://plos.org/resources/writing-center/}. Each journal have a
submission guideline page, please refer to it.

\begin{itemize}
\tightlist
\item
  \href{https://journals.plos.org/plosbiology/s/submission-guidelines}{PLOS
  Biology}
\item
  \href{https://journals.plos.org/climate/s/submission-guidelines}{PLOS
  Climate}
\item
  \href{https://journals.plos.org/digitalhealth/s/submission-guidelines}{PLOS
  Digital Health}
\item
  \href{https://journals.plos.org/ploscompbiol/s/submission-guidelines}{PLOS
  Computational Biology}
\item
  \href{https://journals.plos.org/plosgenetics/s/submission-guidelines}{PLOS
  Genetics}
\item
  \href{https://journals.plos.org/globalpublichealth/s/submission-guidelines}{PLOS
  Global Public Health}
\item
  \href{https://journals.plos.org/plosmedicine/s/submission-guidelines}{PLOS
  Medicine}
\item
  \href{https://journals.plos.org/plosntds/s/submission-guidelines}{PLOS
  Neglected Tropical Diseases}
\item
  \href{https://journals.plos.org/plosone/s/submission-guidelines}{PLOS
  ONE}
\item
  \href{https://journals.plos.org/plospathogens/s/submission-guidelines}{PLOS
  Pathogens}
\item
  \href{https://journals.plos.org/sustainabilitytransformation/s/submission-guidelines}{PLOS
  Sustainability and Transformation}
\item
  \href{https://journals.plos.org/water/s/submission-guidelines}{PLOS
  Water}
\end{itemize}

This template file contains some guidelines and recommandation initially
given in \texttt{plos\_latex\_template.tex} that can be found in
\url{https://github.com/quarto-journals/plos/blob/main/style-guide/plos_latex_template.tex}

\hypertarget{metadata}{%
\subsection{Metadata}\label{metadata}}

\hypertarget{about-journal-id-field}{%
\subsubsection{About journal id field}\label{about-journal-id-field}}

This is an identifier for the target journal. It can be derived from
https://plos.org/resources/writing-center/ following submission
guidelines link, the identifier is the part of the URL after
\texttt{https://journals.plos.org/\textless{}id\textgreater{}/s/submission-guidelines}

\begin{longtable}[]{@{}ll@{}}
\toprule\noalign{}
Journal & id \\
\midrule\noalign{}
\endhead
\bottomrule\noalign{}
\endlastfoot
PLOS Biology & plosbiology \\
PLOS Climate & climate \\
PLOS Digital Health & digitalhealth \\
PLOS Computational Biology & ploscompbiol \\
PLOS Genetics & plosgenetics \\
PLOS Global Public Health & globalpublichealth \\
PLOS Medicine & plosmedicine \\
PLOS Neglected Tropical Diseases & plosntds \\
PLOS ONE & plosone \\
PLOS Pathogens & plospathogens \\
PLOS Sustainability and Transformation & sustainabilitytransformation \\
PLOS Water & water \\
\end{longtable}

Example :

\begin{Shaded}
\begin{Highlighting}[]
\FunctionTok{format}\KeywordTok{:}
\AttributeTok{  }\FunctionTok{plos{-}pdf}\KeywordTok{:}
\AttributeTok{    }\FunctionTok{journal}\KeywordTok{:}
\AttributeTok{      }\FunctionTok{id}\KeywordTok{:}\AttributeTok{ water}
\end{Highlighting}
\end{Shaded}

\hypertarget{once-your-paper-is-accepted-for-publication}{%
\subsection{Once your paper is accepted for
publication,}\label{once-your-paper-is-accepted-for-publication}}

Do not include track change in LaTeX file and leave only the final text
of your manuscript. PLOS recommends the use of latexdiff to track
changes during review, as this will help to maintain a clean tex file.
Visit \url{https://www.ctan.org/pkg/latexdiff?lang=en} for info or
contact us at \href{mailto:latex@plos.org}{\nolinkurl{latex@plos.org}}.

\emph{This should not be a problem using Quarto but still a
recommandation from the journal}

There are no restrictions on package use within the LaTeX files except
that no packages listed in the template may be deleted.

Please do not include colors or graphics in the text. Color can be used
to apply background shading to table cells only.

The manuscript LaTeX source should be contained within a single file (do
not use\texttt{\textbackslash{}input},
\texttt{\textbackslash{}externaldocument}, or similar commands).

Please contact \href{mailto:latex@plos.org}{\nolinkurl{latex@plos.org}}
with any questions submission guidelines. For anything Quarto related,
please open an issue in \url{https://github.com/quarto-journals/plos}.
If this is related to the LaTeX template, this could also be a good idea
to contact PLOS directly.

\hypertarget{figures-and-tables}{%
\subsection{Figures and Tables}\label{figures-and-tables}}

Please include tables/figure captions directly after the paragraph where
they are first cited in the text.

\hypertarget{figures}{%
\subsubsection{Figures}\label{figures}}

However, do not include graphics in your manuscript

\begin{itemize}
\tightlist
\item
  Figures should be uploaded separately from your manuscript file.
\item
  Figures generated using LaTeX should be extracted and removed from the
  PDF before submission.
\item
  Figures containing multiple panels/subfigures must be combined into
  one image file before submission.
\end{itemize}

\textbf{This means that, depending on how you create your figure, a
manual post processing will be required.}

For figure citations, please use ``Fig'' instead of ``Figure''. This has
been made the default in this Quarto format:

\begin{Shaded}
\begin{Highlighting}[]
\FunctionTok{crossref}\KeywordTok{:}
\AttributeTok{  }\FunctionTok{fig{-}title}\KeywordTok{:}\AttributeTok{ Fig }
\end{Highlighting}
\end{Shaded}

Also, place figure captions after the first paragraph in which they are
cited.

See PLOS figure guidelines at
\url{https://journals.plos.org/plosone/s/figures} and in your specific
journal guideline.

\hypertarget{tables}{%
\subsubsection{Tables}\label{tables}}

Tables should be cell-based and may not contain:

\begin{itemize}
\tightlist
\item
  spacing/line breaks within cells to alter layout or alignment
\item
  do not nest tabular environments (no tabular environments within
  tabular environments)
\item
  no graphics or colored text (cell background color/shading OK)
\end{itemize}

See PLOS table guidelines at
\url{http://journals.plos.org/plosone/s/tables} and in your specific
journal guideline.

For tables that exceed the width of the text column, use the adjustwidth
environment as illustrated in the example table in text below. If you
are in this case, you'll either need to manually post process the
\texttt{.tex} file and recreate the PDF, or you need to include LaTeX
tables directly.

Also, place tables after the first paragraph in which they are cited.

\hypertarget{equations-math-symbols-subscripts-and-superscripts}{%
\subsection{Equations, math symbols, subscripts, and
superscripts}\label{equations-math-symbols-subscripts-and-superscripts}}

Below are a few tips to help format your equations and other special
characters according to our specifications. For more tips to help reduce
the possibility of formatting errors during conversion, please see our
LaTeX guidelines at http://journals.plos.org/plosone/s/latex

\begin{itemize}
\item
  For inline equations, please be sure to include all portions of an
  equation in the math environment. For example, \texttt{x\$\^{}2\$} is
  incorrect; this should be formatted as \(x^2\) (or \(\mathrm{x}^2\) if
  the romanized font is desired).
\item
  Do not include text that is not math in the math environment. For
  example, \texttt{CO2} should be written as
  \texttt{CO\textbackslash{}textsubscript\{2\}} giving
  CO\textsubscript{2} instead of \texttt{CO\$\_2\$}.
\item
  Please add line breaks to long display equations when possible in
  order to fit size of the column.
\item
  For inline equations, please do not include punctuation (commas, etc)
  within the math environment unless this is part of the equation.
\item
  When adding superscript or subscripts outside of brackets/braces,
  please group using \texttt{\{\}}. For example, change
  \texttt{"{[}U(D,E,\textbackslash{}gamma){]}\^{}2"} to
  \texttt{"\{{[}U(D,E,\textbackslash{}gamma){]}\}\^{}2"}.\\
\item
  Do not use \texttt{\textbackslash{}cal} for caligraphic font. Instead,
  use \texttt{\textbackslash{}mathcal\{\}}
\end{itemize}

\hypertarget{title-and-headings}{%
\subsection{Title and headings}\label{title-and-headings}}

Please use ``sentence case'' for title and headings (capitalize only the
first word in a title (or heading), the first word in a subtitle (or
subheading), and any proper nouns).

PLOS does not support heading levels beyond the 3rd, meaning no 4th
level headings. Header 4 levels \texttt{\#\#\#\#} is used for the
\emph{Supporting information} section

\hypertarget{abstract-and-author-summary}{%
\subsection{Abstract and author
summary}\label{abstract-and-author-summary}}

Abstract must be kept below 300 words.

Author Summary must be kept between 150 and 200 words and first person
must be used.

For PLOS ONE, author summary won't be included as it is not valid for
submission.

\hypertarget{supplementary-information-syntax}{%
\subsection{Supplementary information
syntax}\label{supplementary-information-syntax}}

Use this markdown syntax to create the supplementary information block
with a custom block of class \texttt{.supp}

\begin{Shaded}
\begin{Highlighting}[]
\NormalTok{::: \{.supp\}}
\FunctionTok{\#\# SI TYPE \{\#id\}}

\NormalTok{First paragraph is a title sentence that will be bold. (required)}

\NormalTok{Optionnaly, add descriptive text after the title of the}
\NormalTok{item. No third paragraph is allowed}
\NormalTok{:::}
\end{Highlighting}
\end{Shaded}

They need to be referenced in text using \texttt{nameref} by using this
syntax \texttt{{[}id{]}(.nameref)} where \texttt{ìd} will be the id used
on the header.

\hypertarget{references}{%
\subsection{References}\label{references}}

Within Quarto, \texttt{natbib} will be used with \texttt{plos2015.bst},
which expect numeric style citation. Use brackets for references, e.g
\texttt{{[}@ref{]}}.

\hypertarget{quarto-features-limitation}{%
\subsection{Quarto features
limitation}\label{quarto-features-limitation}}

Some features are not working with this format in PDF:

\begin{itemize}
\tightlist
\item
  Callouts
\item
  Code highlighting customization (border left, background color)
\end{itemize}

\begin{center}\rule{0.5\linewidth}{0.5pt}\end{center}

\begin{quote}
\textbf{Following content of this document is from the LaTeX template
content to demo journal style.}
\end{quote}

\begin{center}\rule{0.5\linewidth}{0.5pt}\end{center}

\hypertarget{introduction-1}{%
\section{Introduction}\label{introduction-1}}

Lorem ipsum dolor sit~\citep{bib1} amet, consectetur adipiscing elit.
Curabitur eget porta erat. Morbi consectetur est vel gravida pretium.
Suspendisse ut dui eu ante cursus gravida non sed sem. Nullam
Eq~\ref{eq-schemeP} sapien tellus, commodo id velit id, eleifend
volutpat quam. Phasellus mauris velit, dapibus finibus elementum vel,
pulvinar non tellus. Nunc pellentesque pretium diam, quis maximus dolor
faucibus id.~\citep{bib2} Nunc convallis sodales ante, ut ullamcorper
est egestas vitae. Nam sit amet enim ultrices, ultrices elit pulvinar,
volutpat risus.

\begin{equation}\protect\hypertarget{eq-schemeP}{}{
\begin{aligned}
\mathrm{P_Y} = \underbrace{H(Y_n) - H(Y_n|\mathbf{V}^{Y}_{n})}_{S_Y} + \underbrace{H(Y_n|\mathbf{V}^{Y}_{n})- H(Y_n|\mathbf{V}^{X,Y}_{n})}_{T_{X\rightarrow Y}}
\end{aligned}
}\label{eq-schemeP}\end{equation}

\hypertarget{materials-and-methods}{%
\section{Materials and methods}\label{materials-and-methods}}

\hypertarget{etiam-eget-sapien-nibh}{%
\subsection{Etiam eget sapien nibh}\label{etiam-eget-sapien-nibh}}

Nulla mi mi, Fig~\ref{fig1} venenatis sed ipsum varius, volutpat euismod
diam. Proin rutrum vel massa non gravida. Quisque tempor sem et
dignissim rutrum. Lorem ipsum dolor sit amet, consectetur adipiscing
elit. Morbi at justo vitae nulla elementum commodo eu id massa. In vitae
diam ac augue semper tincidunt eu ut eros. Fusce fringilla erat
porttitor lectus cursus, vel sagittis arcu lobortis. Aliquam in enim
semper, aliquam massa id, cursus neque. Praesent faucibus semper libero.

% Place figure captions after the first paragraph in which they are cited.
\begin{figure}[!h]
\caption{{\bf Bold the figure title.}
Figure caption text here, please use this space for the figure panel descriptions instead of using subfigure commands. A: Lorem ipsum dolor sit amet. B: Consectetur adipiscing elit.}
\label{fig1}
\end{figure}

\hypertarget{results}{%
\section{Results}\label{results}}

Results and Discussion can be combined.

Nulla mi mi, venenatis sed ipsum varius, Table~\ref{table1} volutpat
euismod diam. Proin rutrum vel massa non gravida. Quisque tempor sem et
dignissim rutrum. Lorem ipsum dolor sit amet, consectetur adipiscing
elit. Morbi at justo vitae nulla elementum commodo eu id massa. In vitae
diam ac augue semper tincidunt eu ut eros. Fusce fringilla erat
porttitor lectus cursus, \nameref{s1-video} vel sagittis arcu lobortis.
Aliquam in enim semper, aliquam massa id, cursus neque. Praesent
faucibus semper libero.

% Place tables after the first paragraph in which they are cited.
\begin{table}[!ht]
\begin{adjustwidth}{-2.25in}{0in} % Comment out/remove adjustwidth environment if table fits in text column.
\centering
\caption{
{\bf Table caption Nulla mi mi, venenatis sed ipsum varius, volutpat euismod diam.}}
\begin{tabular}{|l+l|l|l|l|l|l|l|}
\hline
\multicolumn{4}{|l|}{\bf Heading1} & \multicolumn{4}{|l|}{\bf Heading2}\\ \thickhline
$cell1 row1$ & cell2 row 1 & cell3 row 1 & cell4 row 1 & cell5 row 1 & cell6 row 1 & cell7 row 1 & cell8 row 1\\ \hline
$cell1 row2$ & cell2 row 2 & cell3 row 2 & cell4 row 2 & cell5 row 2 & cell6 row 2 & cell7 row 2 & cell8 row 2\\ \hline
$cell1 row3$ & cell2 row 3 & cell3 row 3 & cell4 row 3 & cell5 row 3 & cell6 row 3 & cell7 row 3 & cell8 row 3\\ \hline
\end{tabular}
\begin{flushleft} Table notes Phasellus venenatis, tortor nec vestibulum mattis, massa tortor interdum felis, nec pellentesque metus tortor nec nisl. Ut ornare mauris tellus, vel dapibus arcu suscipit sed.
\end{flushleft}
\label{table1}
\end{adjustwidth}
\end{table}

\hypertarget{lorem-and-ipsum-nunc-blandit-a-tortor}{%
\subsection{\texorpdfstring{\textbf{LOREM}~and \textbf{IPSUM}~nunc
blandit a
tortor}{LOREM~and IPSUM~nunc blandit a tortor}}\label{lorem-and-ipsum-nunc-blandit-a-tortor}}

\hypertarget{rd-level-heading}{%
\subsubsection{3rd level heading}\label{rd-level-heading}}

Maecenas convallis mauris sit amet sem ultrices gravida. Etiam eget
sapien nibh. Sed ac ipsum eget enim egestas ullamcorper nec euismod
ligula. Curabitur fringilla pulvinar lectus consectetur pellentesque.
Quisque augue sem, tincidunt sit amet feugiat eget, ullamcorper sed
velit. Sed non aliquet felis. Lorem ipsum dolor sit amet, consectetur
adipiscing elit. Mauris commodo justo ac dui pretium imperdiet. Sed
suscipit iaculis mi at feugiat.

\begin{enumerate}
\def\labelenumi{\arabic{enumi}.}
\item
  react
\item
  diffuse free particles
\item
  increment time by dt and go to 1
\end{enumerate}

\hypertarget{sed-ac-quam-id-nisi-malesuada-congue}{%
\subsection{Sed ac quam id nisi malesuada
congue}\label{sed-ac-quam-id-nisi-malesuada-congue}}

Nulla mi mi, venenatis sed ipsum varius, volutpat euismod diam. Proin
rutrum vel massa non gravida. Quisque tempor sem et dignissim rutrum.
Lorem ipsum dolor sit amet, consectetur adipiscing elit. Morbi at justo
vitae nulla elementum commodo eu id massa. In vitae diam ac augue semper
tincidunt eu ut eros. Fusce fringilla erat porttitor lectus cursus, vel
sagittis arcu lobortis. Aliquam in enim semper, aliquam massa id, cursus
neque. Praesent faucibus semper libero.

\begin{itemize}
\item
  First bulleted item.
\item
  Second bulleted item.
\item
  Third bulleted item.
\end{itemize}

\hypertarget{discussion}{%
\section{Discussion}\label{discussion}}

Nulla mi mi, venenatis sed ipsum varius,see Table~\ref{table1} volutpat
euismod diam. Proin rutrum vel massa non gravida. Quisque tempor sem et
dignissim rutrum. Lorem ipsum dolor sit amet, consectetur adipiscing
elit. Morbi at justo vitae nulla elementum commodo eu id massa. In vitae
diam ac augue semper tincidunt eu ut eros. Fusce fringilla erat
porttitor lectus cursus, vel sagittis arcu lobortis. Aliquam in enim
semper, aliquam massa id, cursus neque. Praesent faucibus semper
libero~\citep{bib3}.

\hypertarget{conclusion}{%
\section{Conclusion}\label{conclusion}}

CO\textsubscript{2} Maecenas convallis mauris sit amet sem ultrices
gravida. Etiam eget sapien nibh. Sed ac ipsum eget enim egestas
ullamcorper nec euismod ligula. Curabitur fringilla pulvinar lectus
consectetur pellentesque. Quisque augue sem, tincidunt sit amet feugiat
eget, ullamcorper sed velit.

Sed non aliquet felis. Lorem ipsum dolor sit amet, consectetur
adipiscing elit. Mauris commodo justo ac dui pretium imperdiet. Sed
suscipit iaculis mi at feugiat. Ut neque ipsum, luctus id lacus ut,
laoreet scelerisque urna. Phasellus venenatis, tortor nec vestibulum
mattis, massa tortor interdum felis, nec pellentesque metus tortor nec
nisl. Ut ornare mauris tellus, vel dapibus arcu suscipit sed. Nam
condimentum sem eget mollis euismod. Nullam dui urna, gravida venenatis
dui et, tincidunt sodales ex. Nunc est dui, sodales sed mauris nec,
auctor sagittis leo. Aliquam tincidunt, ex in facilisis elementum,
libero lectus luctus est, non vulputate nisl augue at dolor. For more
information, see \nameref{s1-appendix}.

\hypertarget{supporting-information}{%
\section{Supporting information}\label{supporting-information}}

\paragraph*{S1 Fig.}
\label{s1-fig}
{\textbf{Bold the title sentence.}} Add descriptive text after the title
of the item (optional).

\paragraph*{S2 Fig.}
\label{s2-fig}
{\textbf{Lorem ipsum.}} Maecenas convallis mauris sit amet sem ultrices
gravida. Etiam eget sapien nibh. Sed ac ipsum eget enim egestas
ullamcorper nec euismod ligula. Curabitur fringilla pulvinar lectus
consectetur pellentesque.

\paragraph*{S1 File.}
\label{s1-file}
{\textbf{Lorem ipsum.}}

\paragraph*{S1 Video.}
\label{s1-video}
{\textbf{Lorem ipsum.}} Maecenas convallis mauris sit amet sem ultrices
gravida. Etiam eget sapien nibh. Sed ac ipsum eget enim egestas
ullamcorper nec euismod ligula. Curabitur fringilla pulvinar lectus
consectetur pellentesque.

\paragraph*{S1 Appendix.}
\label{s1-appendix}
{\textbf{Lorem ipsum.}} Maecenas convallis mauris sit amet sem ultrices
gravida. Etiam eget sapien nibh. Sed ac ipsum eget enim egestas
ullamcorper nec euismod ligula. Curabitur fringilla pulvinar lectus
consectetur pellentesque.

\paragraph*{S1 Table.}
\label{s1-table}
{\textbf{Lorem ipsum.}} Maecenas convallis mauris sit amet sem ultrices
gravida. Etiam eget sapien nibh. Sed ac ipsum eget enim egestas
ullamcorper nec euismod ligula. Curabitur fringilla pulvinar lectus
consectetur pellentesque.

\hypertarget{acknowledgments}{%
\section{Acknowledgments}\label{acknowledgments}}

Cras egestas velit mauris, eu mollis turpis pellentesque sit amet.
Interdum et malesuada fames ac ante ipsum primis in faucibus. Nam id
pretium nisi. Sed ac quam id nisi malesuada congue. Sed interdum aliquet
augue, at pellentesque quam rhoncus vitae.

zzzzz


\nolinenumbers
  \bibliography{references.bib}

\end{document}
